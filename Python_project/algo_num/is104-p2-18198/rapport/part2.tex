\section{Analyse des résultats}

\subsection{Factorisation complète}

\subsubsection{Factorisation dense de Cholesky.}

La complexité de notre algorithme pour résoudre la factorisation dense de Choleski est en \textit{O(n³)}.

\vspace{0.25cm}

Le cout d'une résolution de système linéaire dense \textit{\textbf{A.x = b}} est en \textit{O(n²)}.

\subsection{Factorisation de Cholesky incomplète}

\subsubsection{Génération de matrices symétriques}

\subsubsection{Algorithme de factorisation de Choleski incomplète}

La complexité de notre algorithme pour résoudre la factorisation dense de Choleski est en \textit{O(n³)}

Pour vérifier cette implémentation, nous avons généré à l'aide d'une fonction de test des matrices symétriques auxquelles nous avons appliqué la factorisation . Nous avans ensuite testé qu'elles étaient bien triangulaires inférieures, que lors d'une multiplication avec sa transposée, nous obtenions bien l'original, et qu'elle était bien définie positive.

\subsection{Méthode du gradient conjugué}

\subsubsection{Sans préconditionneur}

La complexité de notre algorithme pour implémenter la méthode du gradiant conjugué est en \textit{O(n²)}.

\subsubsection{Avec préconditionneur}

La complexité de notre algorithme pour implémenter la méthode du gradiant conjugué est en \textit{O(n²)}.

Malheureusement, après une série de tests similaires à ceux de la version sans préconditionneur, les tests ne se sont pas avérés concluants.

\subsection{Application à l'équation de la chaleur}

Afin de résoudre le problème precedemment posé nous avons implémenté plusieurs fonctions en python.
\\
\\
D'abord pour résoudre le problème en utilsant la décomposoiton de Choleski et la méthode du gradient conjugué, puis en utilisant leurs méthodes homologues incomplètes.
\\
\\
Il a d'abord fallu déclarer une matrice de temperature, initialiser ses conditions limites, et choisir une fonction de la source de châleur.

\subsubsection{Décomposition de Cholesky et gradient conjugué}
\subsubsection{Décomposition de Cholesky incomplète et gradient conjugué incomplet}
